\section{Methodology}

We tested various devices against our 802.11ac WAP in both clear and RF-noisy
environments. We originally planned to measure the differences between several
different MCSs, on both 80MHz and 40MHz, and the effects of MU-MIMO and
beamforming on a connection. Unfortunately, due to hardware limitations, we were
only able to test 80 MHz vs 40MHz, and monitor some beamforming and MCS activity
without influencing or selecting it.

\subsection{Devices}

\paragraph{RF Test Equipment}

We used a spectrum analyzer that is capable of monitoring 2.4GHz and 5GHz WiFi
signals allowing us to see all RF traffic in the spectrums beyond just WiFi.
We used it to observe active networks and to confirm that our clear testing
location had minimal to no RF noise.

\paragraph{Wireless APs}

Our WAP is the latest hardware model of the Linksys WRT1900AC 802.11ac,
with a 1.2 GHz dual core ARM processor and 256MB of DDR3 RAM. We originally
installed OpenWRT Chaos Calmer (Dev Branch) on the AP, because we believed it
would give us complete configuration control and insight into what the AP is
doing. Unfortunately, after many tests, we determined that bugs or
inefficiencies in the OpenWRT firmware were causing issues with our throughput
numbers, and we restored the original Linksys stock firmware to the device.
After doing so, our results returned to the expected levels for throughput.

\paragraph{Clients}

We used a number of devices as 802.11ac clients, including an Apple Macbook Pro,
a Microsoft Surface Pro 3, and a Lenovo \todo{???} with a Intel \todo{???} PCIe
802.11ac capable wifi card installed.

We also purchased an ASUS USB 801.11ac dongle so that we could utilize any other
non-ac host as an ac host for testing. However after a small number of tests,
the ASUS dongle failed.

\subsection{Testing}

\paragraph{Setup}
To run our tests, we connected one computer via ethernet cable to the AP, and
then connected the device we were testing over 802.11ac wireless. We would then
run an iperf version 2 \todo{more specific} server on the wireless client, and
an iperf client on the hardwired device. This way the flow of data over wifi
would be from the router to the wireless client. For a majority of our tests, we
did four transfers of 250MB, though for some tests, we did one transfer lasting
100 seconds. For a majority of tests, we captured packets either on the device
doing the transfer itself, or on a nearby device using monitor mode. This way
we could get high level throughput information from iperf, and then examine the
packet captures afterwards to help understand and explain our results.

\paragraph{Locations}
We wanted to perform our tests in both RF-clear and RF-noisy environments. For
the noisy environment, we used our offices in the CSE building, which are
surrounding by 802.11b/g/n/ac APs and devices. For the clear environment, we
performed initial promising tests in the basement level of the Hopkins parking
structure, however we were concerned that the low concrete beams in the ceiling
could be inadvertently effecting our results. We then moved out to the parking
lot behind the Torrey Pines glider port. Here we had a wide open space with no
RF-interference according to our spectrum analyzer.


\paragraph{Indoor}
In the office, we set the WAP at a fixed point, and then measured out 5, 10 and
15ft increments from the device for initial tests. We then moved our clients out
into the hallway, so the signal had to travel approximately 19 feet and through
a single wall. Then we measured from the kitchen, so the signal had to travel
through two walls for a total distance traveled of approximately 23 feet, and
then on the other side of the building, approximately 51 feet away, so that the
signal had to travel through several walls and devices. We tested each of our
devices with the WAP set to both 80MHz and 40MHz.  However, we were unable to
force the WAP or any of the clients to stick to a single MCS, and we were unable
to disable or influence beamforming on any of the clients.

\paragraph{Clear}
Out at the glider port, we measured out several different distances from the
router and ran tests from there. Because of the size of the parking lot, we were
able to run tests much further away from the router here than in the office.

\paragraph{ \todo{title this better} Things we measured }
We tested each of our devices with the WAP set to both 80MHz and 40MHz.
However, we were unable to force the WAP or any of the clients to stick to a
single MCS, and we were unable to disable or influence beamforming on any of the
clients. Because of this, we were only able to observe the effects of
beamforming or different MCSs as they normally occurred.

\todo{ not done yet }

\subsection{MU-MIMO Testing}

We also originally intended to measure the effect of MU-MIMO on throughput and
packet drops, however we ran into some complications. Of the three chipset
drivers that support MU-MIMO, none have any products that are currently on the
market \todo{david, is this sentence correct?}. So although our WAP supported
MU-MIMO, no clients exist that support MU-MIMO.

Despite this, we set up what we believe is a strong experimental design to test
MU-MIMO and confirm that it is occurring through monitoring the clients. The
setup is as follows:

There are two wireless connected laptops running iperf servers, as in the
previously described setup. These two clients should be placed on opposite sides
or orthogonal to the router at some fixed distance. Near these laptops, two more
laptops with their wireless cards placed in monitor mode will log all the
packets they receive.

Then, connected via ethernet to the AP will be two iperf clients, so that data
flow travels from the router to the wireless clients. A third device will be
connected to the AP via ethernet which broadcasts an increasing index and a
64-bit timestamp every 100 milliseconds. This way, we can correlate runs of
packets between broadcasts with matching indexes and monitor clock skew, similar
to \todo{cite ucsd wifi paper}.

We believe that, given proper MU-MIMO hardward, this testing setup would be
capable of confirming that MU-MIMO is indeed happening, and allowing us to
reason about its effects on the network.



