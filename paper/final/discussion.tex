\section{Discussion}

The surface was seen to outperform the other systems in the indoor test at the furthest location. These results did not make any sense to us so we did some investigation into the packet captures. The Surface consistently sent at MCS indices 4 and 5 using two spatial streams for a theoretical data rate of 490 Mbps. The Intel card always sent at MCS index 2 with only 1 spatial stream for a theoretical data rate of 83 Mbps. This accounts for the difference in throughput between these two devices however the Mac was a special case. It used an MCS of 4 and 5 like the Surface but used 2 and 3 spatial streams consistently. However, the performance was hindered drastically by the amount of TCP retransmits that it received. Unfortunately we did not capture on the AP but we believe that the ACKs were not making it back to the AP when the Mac transmitted them. There were no retransmits at the 802.11 MAC layer so this indicated that the packet was resent by the other endpoint based on a timeout and the ACK packets sent by the Mac were unfortunately not captured to verify this behavior. The Mac also had performance issues when capturing its own packets and this appears to affect the throughput and timing such as turn around time for sending ACKs. One other peculiarity that both the Mac and Intel exhibited was the receive window size keep growing to a large value, where as on the Surface this value remained at 64K at all times. Once again, we did not capture packets at the other endpoint or AP so we were not able to resolve why this behavior was occurring exactly.

In the beginning we started to use iPerf version 3.0.11 but quickly discovered that it had some bugs/peculiar operations. When using iPerf3 for our testing we would always get higher throughput when the wireless client sent to the wired client. This was counter-intuitive to us, so we did some investigation and eventually tried iPerf 2 to see if this behavior was the same. It was not the same, it was as we expected, that the other direction (wired client to wireless client) was always a greater throughput.

We also started to use OpenWRT Chaos Calmer (Dev. branch) on the Linksys WAP so that we could have more flexibility, control, and insight into what the WAP was doing. However, after consistently experiencing poor performance of the WiFi network, we resorted back to the latest stock firmware to determine if the hardware or firmware was the cause. Once we had the stock firmware back in place, the performance of the WiFi network was as we expected. At this point, we stuck with the stock firmware and adjusted our testing framework to account for the lack of control at the AP where we could.
