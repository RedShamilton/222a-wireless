\section{background}

Today we see the now ubiquitous WiFi networks in almost every setting where computers are used from home and businesses to public WiFi hotspots. This trend has grown over the years as mobile devices such as smart phones require high-speed network access and laptops are frequently reducing their port count eliminating wired network options.

The IEEE 802.11 standard for wireless Ethernet has evolved over the years to keep pace with these ever increasing demand for higher throughput wireless network connectivity.

\subsection{Previous Standards}

There were four widely used revisions of 802.11 before ac; b, g, a, and n. The first widely used revision, b, was introduced in September 1999 \cite{ieee802b} and is still seen in use today in many lower-end devices. This revision operated in the 2.4GHz spectrum and made use of eleven 22MHz channels in the US (and up to 14 in other countries). The b revision uses Direct-Sequence Spread Spectrum (DSSS) as the modulation scheme dropping the Frequency-Hopping Spread Spectrum (FHSS) that was used in legacy versions of 802.11 prior to the release of b. At the same time that b was released the a revision a \cite{ieee802a} was also released replacing DSSS with Orthogonal Frequency Division Multiplexing (OFMD) as the modulation scheme and changing from 22MHz to 20MHz bandwidth channels in the 5GHz spectrum. This allowed for data rates of up to 54Mbps over the 11Mbps that b was capable of. The next standard, g, brought the modulation schemes and 20MHz channel bandwidth of revision a over to the 2.4GHz spectrum while maintaining backwards compatibility by supporting modulation schemes and channel bandwidths of b \cite{ieee802g}.

The next revision, n (which ac builds upon and simplifies further), aimed to consolidate technologies used on both 2.4GHz and 5GHz while obsoleting outdated parts of the standard and introducing a few new capabilities that worked on both spectrums to increase throughput and efficiency \cite{ieee802n}. The n standard adds three main capabilities to the 802.11 standard; Multi-Input Multi-Output (MIMO) and 40 MHz bandwidth channels to the PHY layer and frame aggregation to the MAC layer. MIMO two key features that increase throughput and reliability of 802.11 wireless networks; namely spatial beamforming and multiplexing. Spatial beamforming is used to increase signal strength at the receiver using coding techniques and multiple transmit antennas to transmit a signal from each antenna with a phase offset such that the superposition of the RF waves is higher at the receiver than what a omnidirectional transmit would achieve. Spatial multiplexing makes use of the multiple transmit antennas of the transmitter and multiple receive antennas of the receiver to send  simultaneous signals on the same channel separated spatially by the physical arrangement of antennas used to transmit that stream of data. This is referred to as Spatial Division Multiplexing and is related to how OFDM divides the channel into sub channels by frequency expect it divides the channel up spatially. These divisions (or sub channels) are referred to as spatial streams and are achieved using coding techniques at both the sender and receiver. One difference to note between these two distinct MIMO enabled technologies is that spatial beamforming only requires multiple transmit antennas to achieve whereas spatial multiplexing requires the same amount of receive antennas as transmit antennas on the two communicating devices to achieve spatial streams. The number of spatial streams between any two devices is the lowest of their transmit and receive chains (i.e. antennas). It is also worth noting that MIMO does not come for free as there must be a separate transmit and receive chain for each antenna which increases area, power consumption, and the price of these devices over previous revisions. The 40MHz bandwidth channels that revision n adds increases throughput by enabling more OFDM subcarriers per channel leading to slightly more than double data rates that are achievable. The frame aggregation that revision n adds increases useable throughput by grouping multiple MAC Protocol or Service Data Units (MPDU or MSDU) into a single frame to amortize the overhead cost of transmitting a these each in a separate frame. Frame aggregation is accompanied by block acknowledgements, a special control frame that can acknowledge up to 64 received frames at once.

\begin{table*}[bth]
\caption{Summary of IEEE 802.11 Revisions}
\begin{center}
\begin{tabular}{c | c | c | c | p{1.3cm} | c}
\textbf{revision} & \textbf{release date} & \textbf{spectrum (GHz)} & \textbf{bandwidth (MHz)} & \textbf{modulation} & \textbf{spatial streams} \\
\hline\hline
a & 9/1999 & 5 & 20 & OFDM & N/A \\ \hline
b & 9/1999 & 2.4 & 22 & CCK\newline DSSS & N/A \\ \hline
g & 6/2003 & 2.4 & 20 & CCK\newline OFDM\newline DSSS & N/A \\ \hline
n & 10/2009 & 2.4/5 & 20/40 & OFDM & 4 \\ \hline
ac & 12/2013 & 5 & 20/40/80/160 & OFDM & 8 \\ \hline
\end{tabular}
\end{center}
\label{table:80211caps}
\end{table*}

\subsection{IEEE 802.11ac}

The latest 802.11 revision that is now available in many consumer grade products is revision ac. This revision is a direct extension to the n standard and brings increased throughput and reliability to wireless LANs on the 5GHz spectrum.

\subsubsection{Changes to Previous Standards}

802.11ac has cleaned up and standardized some key features introduced in revision n. The first feature that has been standardized is beamforming. With revision n beamforming had many different methods to accomplish the same tasks and often different manufacturers chose to implement only some of these available methods making capability between different brands difficult or unlikely. The next feature that ac has revised is how the Modulation and Coding Scheme (MCS) indices are defined. The notation of MCS was introduced in revision n and is a way of conveying the data rate between devices. However, in revision n there were over 30 indices with much redundancy in the information they contained. The MCS for revision n encoded the modulation technique, coding scheme, and number of spatial streams. Revision n had up to 4 spatial streams, meaning the same set of data was repeated 4 times for in the MCS indices with only the number of spatial streams being different. In 802.11ac there are only 10 MCS indices which indicate only the modulation technique and coding scheme and the number of spatial streams is in its own field now. The maximum number of spatial streams was also increased from 4 to 8 in this revision.

\subsubsection{802.11ac Specific Features}

There are only a handful of new features that are added by the ac revision of the standard but one of these features is potentially a ground breaking feature for WiFi and may change the picture of wireless networking for the better. The first addition is the mandatory 80 MHz channel bandwidth requirement with the optional 160MHz as part of the standard \ref{ieee802ac}. This increased bandwidth increases throughput in the same that is described above for the n revision. The revision also adds QAM256 modulation scheme beyond the QAM64 supported by n. This increases the symbol rate per subcarrier (and overall throughput) but requires a higher SNR to demodulate as a trade off. Perhaps the most important feature that ac adds is Multi-User MIMO (MU-MIMO). MU-MIMO is in contrast to single-user MIMO (SU-MIMO) which considers a single multi-antenna transmitter communicating with a single multi-antenna receiver making use of spatial streams and beamforming. The multi-user variant utilizes the combination of spatial streams and beamforming to achieve a single multi-antenna transmitter communicating with multiple multi-antenna receivers in parallel. This is achieved by splitting up the total number of spatial streams of the transmitter to multiple receivers. This analogous to the difference between an Ethernet hub and switch, where the latter splits up broadcast domains and allows for simultaneous connections between systems. MU-MIMO does this same thing but on a much smaller scale currently but future iterations could realize an actual wireless switch.
