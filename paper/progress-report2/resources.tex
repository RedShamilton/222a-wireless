\section{Resources}

\subsection{RF Test Equipment}

We have a spectrum analyzer that is capable of monitoring 2.4GHz and
5GHz WiFi signals allowing us to see all RF traffic in there spectrums
beyond just WiFi. We are using it to observe active networks and to
confirm that our testing location have minimal to no RF noise.

We've used this to 'test the waters' in several locations, including
the CSE building, the basement of the Hopkins parking structure, and
the gliderport parking lot. The latter two are nearly noise free, so
we have considered them to be reasonable baseline testing locations.

\subsection{Wireless APs}

We are currently using the latest hardware model of the Linksys
WRT1900AC 802.11ac WAP, with a 1.2 GHz dual core ARM processor and
256MB of DDR3 RAM. It is running OpenWRT Chaos Calmer(Dev Branch). The
OpenWRT platform gives us reasonable configuration control and insight
into what the AP is doing. We can configure the bandwidth that the
802.11ac channels use and specify the data rates that are available,
which correlate to the MCS numbers that are used.

\subsection{Clients}

We have a number of devices that can act as 802.11ac clients,
including an Apple Macbook Pro, a Microsoft Surface Pro 3, and two
mobile devices, an Apple iPhone 6+ and an Android OnePlus One.

The MacBook Pro is capable of entering monitor mode and supports the RadioTap
headers which allows us to log all packets that the laptop sees (including other WAPs)
and gives vital information such as S/N, channel frequency and bandwidth, data rate (MCS),
and various other fields.

We have also purchased an ASUS USB 801.11ac dongle so that we can
utilize any other non-ac host as an ac host for testing. We have
inspected the driver code and options for this device, and have built
several versions of the driver, disabling some features and adding
additional logging.

Additionally, we have  purchased an Intel 7260 chipset based mini PCIe
card, which is one of the only available chipsets that supports
monitor mode with its Linux drivers.
