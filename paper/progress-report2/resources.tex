\section{Resources}

\subsection{RF Test Equipment}

We have a spectrum analyzer that is capable of monitoring 2.4GHz and
5GHz WiFi signals allowing us to see all RF traffic in there spectrums
beyond just WiFi. We are using it to observe active networks and to
confirm that our testing location have minimal to no RF noise.

We've used this to 'test the waters' in several locations, including
the CSE building, the basement of the Hopkins parking structure, and
the gliderport parking lot. The latter two are nearly noise free, so
we have considered them to be reasonable baseline testing locations.

\subsection{Wireless APs}

We are currently using the latest hardware model of the Linksys
WRT1900AC 802.11ac WAP, with a 1.2 GHz dual core ARM processor and
256MB of DDR3 RAM. It is running OpenWRT Chaos Calmer(Dev Branch) with
various tools, such as iperf3 (see tools) directly on the router for
testing. The OpenWRT platform gives us complete configuration control
and insight into what the AP is doing. Most importantly, we have
confirmed that we can enable/disable beamforming, which will allow us
to ensure that we are testing its effects.

\subsection{Clients}

We have a number of devices that can act as
802.11ac clients, including an Apple Macbook Pro, a Microsoft Surface Pro 3,
and two mobile devices, an Apple iPhone 6+ and an Android OnePlus
One.

We have also purchased an ASUS USB 801.11ac dongle so that we can
utilize any other non-ac host as an ac host for testing. We have
inspected the driver code and options for this device, and have built
several versions of the driver, disabling some features and adding
additional logging.

We have addtionally purchased an Intel 7260 chipset based mini PCIe
card, which is one of the only available chipsets that supports
monitor mode.

\todo{we should say more here}
